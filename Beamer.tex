\documentclass{beamer}
\usepackage[utf8]{inputenc}
\usepackage{graphicx}
%\graphicspath{ {abcd} }
\title{Simulation of a 4 Stroke Internal Combustion Engine}
\subtitle{A CS251 Project By Group 19 (Inficoders)}
\author{Naveen Kumar \\ 140050013 knaveen@cse.itb.ac.in  \\ Yathansh Kathuria \\ 140050021 yathansh.kathuria@cse.iitb.ac.in \\ Rajat Chaturvedi \\ 140050027 chaturvedirajat96@cse.itb.ac.in}

 
\begin{document}
 
\frame{\titlepage}
 
\begin{frame}{About the Project}

 This simulation show the working of a 4 Stroke Internal Combustion Engine, 
 explaining all its 4 stokes clearly.\\
\begin{enumerate}
\item[$\bullet$]Intake Stroke\\
\item[$\bullet$]Compression Stroke\\
\item[$\bullet$]Power Stroke\\
\item[$\bullet$]Exhaust Stroke\\
\end{enumerate}
It also shows the working of a basic gear transmission system, and how different speeds and power can be achieved using different combinations


\end{frame}
\section{Introduction}
\begin{frame}{Technical specifications and Challenges faced}
The various parts involved in the structure are:\\
\begin{enumerate}
\item[$\bullet$]The main piston system\\
\item[$\bullet$]The left and right valve systems\\
\item[$\bullet$]The supporting structures (The enclosing stands and chambers)\\
\item[$\bullet$]The inlet/outlet chambers and air molecule\\
\item[$\bullet$]The gears\\
\end{enumerate}
Challenges faced:\\
\begin{enumerate}
\item[$\bullet$]Using Concave objects: Had to be broken down into smaller convex ones.
\item[$\bullet$]Time lag in BOX-2D to detect collisions
\item[$\bullet$]Transporting air molecules from outlet to the inlet
\end{enumerate}
\end{frame}
 \section{ Examples }

 
 \subsection{subsection 1}
 \begin{frame}{Our Team}
 Our team is a perfect example of working together and collaboration. As such the project saw equal contributions from all the three group members. But broadly classifying this is the work each one of us did: 
 \begin{enumerate}
 \item[$\bullet$]Naveen Kumar: Handled most of the coding work
 \item[$\bullet$]Yathansh Kathuria: Came up with the design and structure of the engine and worked alongside Naveen to write the code 
 \item[$\bullet$]Rajat Chaturvedi: Was like the head, debugged the codes, made changes in the design so that it was feasible in BOX-2D
 \end{enumerate}
 link to youtube video \url{https://www.cse.iitb.ac.in/~knaveen/box2d.html}
 \end{frame}

 \subsection{subsection2}
\begin{frame}{Citations}
\begin{thebibliography}{9}
\bibitem{ref1}
\url{https://www.youtube.com/watch?v=8kZRpouZ3OQ}

\bibitem{ref2}
\url{http://www.box2d.org/manual.html}

\bibitem{ref3}
\url{http://www.iforce2d.net/b2dtut/}

\bibitem{ref4}
\url{http://codingowl.com/readblog.php?blogid=119}

\bibitem{ref5}
\url{http://github.com/}

\bibitem{ref6}
\url{https://www.stack.nl/~dimitri/doxygen/manual/}

\bibitem{ref8}
\url{https://www.sharelatex.com/learn/Bibliography_management_with_bibtex}

\bibitem{ref9}
\url{https://www.atlassian.com/git/tutorials/undoing-changes/git-checkout}

\bibitem{ref12}
\url{w3school.com}

\bibitem{ref13}
\url{https://apps.ubuntu.com/cat/applications/kazam/}


\end{thebibliography}

 \end{frame}
\end{document}
